\documentclass{article}
\usepackage{framed, color}
\usepackage{textpos}
\usepackage{natbib}
\usepackage[top=1in, bottom=1in, left=1in, right=1in]{geometry}
\usepackage{color}
\usepackage{hyperref}
\usepackage{textcomp}
\usepackage{graphicx}
\usepackage{fancybox}
\usepackage{setspace}
\hypersetup{colorlinks=false, urlcolor=blue, citecolor=black}
\usepackage{soul}
%\onehalfspacing
\usepackage{geometry}
\usepackage{color}
\newgeometry{top=1in, bottom=1in, left=1in, right=1in}
\usepackage{fancyhdr}
\usepackage{wrapfig}
\usepackage{mdframed}
\pagenumbering{arabic}

\usepackage{fontspec}
\setmainfont{Times New Roman}


\begin{document}
\parindent 0.000000001in
\setcounter{page}{0}
\pagenumbering{arabic}

\fancyhead[CO]{Matthew D. MacManes | Project Description}
\pagestyle{fancy}




\large{\textbf{\textsc{Overview:}}} \\
\normalsize 
The maintenance of water balance in animals is one of the most important physiologic processes, and is critical to desert survival. Indeed, mammals are exquisitely sensitive to changes in osmolality, with slight derangement eliciting physiologic compromise.  When the loss of water exceeds dietary intake, dehydration -- and in extreme cases, death -- can occur.  Unlike most mammals, animals living in desert habitats are subjected to long periods of extreme heat and intense drought.  As a result, desert animals have evolved mechanisms through which physiologic homeostasis is maintained despite severe and prolonged dehydration. \textbf{The proposed research uses a novel approach integrating physiology, evolutionary genomics, and computational biology to better understand how animals survive in what appear to be unsurvivable conditions.} Specifically, I study the desert rodent \textit{Peromyscus eremicus} in a captive setting using a walk-in desert chamber where all relevant environmental conditions can be manipulated. I will collect data on the physiologic, metabolomic, and genomic response to extreme heat and aridity in order to identify the key physiologic mechanisms that have evolved to enhance survival. \\

\large{\textbf{\textsc{Intellectual Merits:}}} \\
\normalsize
The study of ecological adaptation, or the process through which animals become fitted to their environment has intrigued researchers for decades, though only recently have we had the ability to study the underlying genomic mechanisms. One particularly salient example of the connection between studies of ecology, natural history and modern genomics can be found in the study of physiologic adaptation to desert conditions. Here, remarkable physiologic, morphologic or behavioral adaptation has been studied in the context of desert ecology, but the elucidation of the connections between physiologic and genomic processes is only now possible. \ul{The proposed research aims to generate a uniquely rich dataset, integrating cutting edge genomic techniques with careful characterization of physiology, all within an ecological context of desert life.}  It will significantly enhance our understanding of the physiologic processes underlying osmoregulation in extreme environments. This project both leverages and enhances existing ecological, behavioral, and genomic resources available to an active \textit{Peromyscus} research community, ultimately allowing for insightful studies of comparative functional biology not possible in traditional model organisms like \textit{Mus}.\\

\large{\textbf{\textsc{Broader Impacts:}}} \\
\normalsize
As a scientist of Native American descent, I strive to be a role model and mentor for underrepresented groups. To this end, I am developing an internship program that aims to expose Native American students to genomics and informatics. This internship supplements my general interest in and track record of recruiting students from underrepresented groups. I aim to \textbf{broadly disseminate} my work via a website, blog, and social media. All publications are hosted on a public preprint server, and published using open access options. All data and code is freely available using public repositories like Github, Dryad, or Figshare. In addition to the intellectual merits listed above, there are significant \textbf{benefits to society}. First, Earth is becoming both warmer and drier, and accurate prediction of animals' response to a changing climate requires knowledge of how animals currently living in dry/hot climates survive -- this project will deepen the requisite knowledge. Next, kidney disease effects millions of Americans. Though the causes are diverse, the pathophysiology of many resemble dehydration (\textit{e.g.} low renal perfusion pressure). Thus, having a better understanding of how desert rodents endure water stress seemingly without complication may enhance our ability to explain why humans and most other mammals cannot. Last, the development of \textit{P. eremicus} as a model system for manipulative functional genomic study of desert biology significantly \textbf{enhances research infrastructure} as a mammalian model of extreme physiologic water conservation does not exist.   

\\  



































\end{document}
